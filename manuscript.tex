% Options for packages loaded elsewhere
\PassOptionsToPackage{unicode}{hyperref}
\PassOptionsToPackage{hyphens}{url}
%
\documentclass[
]{article}
\usepackage{amsmath,amssymb}
\usepackage{lmodern}
\usepackage{iftex}
\ifPDFTeX
  \usepackage[T1]{fontenc}
  \usepackage[utf8]{inputenc}
  \usepackage{textcomp} % provide euro and other symbols
\else % if luatex or xetex
  \usepackage{unicode-math}
  \defaultfontfeatures{Scale=MatchLowercase}
  \defaultfontfeatures[\rmfamily]{Ligatures=TeX,Scale=1}
\fi
% Use upquote if available, for straight quotes in verbatim environments
\IfFileExists{upquote.sty}{\usepackage{upquote}}{}
\IfFileExists{microtype.sty}{% use microtype if available
  \usepackage[]{microtype}
  \UseMicrotypeSet[protrusion]{basicmath} % disable protrusion for tt fonts
}{}
\makeatletter
\@ifundefined{KOMAClassName}{% if non-KOMA class
  \IfFileExists{parskip.sty}{%
    \usepackage{parskip}
  }{% else
    \setlength{\parindent}{0pt}
    \setlength{\parskip}{6pt plus 2pt minus 1pt}}
}{% if KOMA class
  \KOMAoptions{parskip=half}}
\makeatother
\usepackage{xcolor}
\IfFileExists{xurl.sty}{\usepackage{xurl}}{} % add URL line breaks if available
\IfFileExists{bookmark.sty}{\usepackage{bookmark}}{\usepackage{hyperref}}
\hypersetup{
  hidelinks,
  pdfcreator={LaTeX via pandoc}}
\urlstyle{same} % disable monospaced font for URLs
\usepackage[margin=1in]{geometry}
\usepackage{longtable,booktabs,array}
\usepackage{calc} % for calculating minipage widths
% Correct order of tables after \paragraph or \subparagraph
\usepackage{etoolbox}
\makeatletter
\patchcmd\longtable{\par}{\if@noskipsec\mbox{}\fi\par}{}{}
\makeatother
% Allow footnotes in longtable head/foot
\IfFileExists{footnotehyper.sty}{\usepackage{footnotehyper}}{\usepackage{footnote}}
\makesavenoteenv{longtable}
\usepackage{graphicx}
\makeatletter
\def\maxwidth{\ifdim\Gin@nat@width>\linewidth\linewidth\else\Gin@nat@width\fi}
\def\maxheight{\ifdim\Gin@nat@height>\textheight\textheight\else\Gin@nat@height\fi}
\makeatother
% Scale images if necessary, so that they will not overflow the page
% margins by default, and it is still possible to overwrite the defaults
% using explicit options in \includegraphics[width, height, ...]{}
\setkeys{Gin}{width=\maxwidth,height=\maxheight,keepaspectratio}
% Set default figure placement to htbp
\makeatletter
\def\fps@figure{htbp}
\makeatother
\setlength{\emergencystretch}{3em} % prevent overfull lines
\providecommand{\tightlist}{%
  \setlength{\itemsep}{0pt}\setlength{\parskip}{0pt}}
\setcounter{secnumdepth}{5}
\newlength{\cslhangindent}
\setlength{\cslhangindent}{1.5em}
\newlength{\csllabelwidth}
\setlength{\csllabelwidth}{3em}
\newlength{\cslentryspacingunit} % times entry-spacing
\setlength{\cslentryspacingunit}{\parskip}
\newenvironment{CSLReferences}[2] % #1 hanging-ident, #2 entry spacing
 {% don't indent paragraphs
  \setlength{\parindent}{0pt}
  % turn on hanging indent if param 1 is 1
  \ifodd #1
  \let\oldpar\par
  \def\par{\hangindent=\cslhangindent\oldpar}
  \fi
  % set entry spacing
  \setlength{\parskip}{#2\cslentryspacingunit}
 }%
 {}
\usepackage{calc}
\newcommand{\CSLBlock}[1]{#1\hfill\break}
\newcommand{\CSLLeftMargin}[1]{\parbox[t]{\csllabelwidth}{#1}}
\newcommand{\CSLRightInline}[1]{\parbox[t]{\linewidth - \csllabelwidth}{#1}\break}
\newcommand{\CSLIndent}[1]{\hspace{\cslhangindent}#1}
\ifLuaTeX
  \usepackage{selnolig}  % disable illegal ligatures
\fi

\author{}
\date{\vspace{-2.5em}}

\begin{document}

{
\setcounter{tocdepth}{2}
\tableofcontents
}
\hypertarget{titulo-do-projeto}{%
\section{TITULO DO PROJETO}\label{titulo-do-projeto}}

\textbf{ABSTRACT}:

\textbf{KEYWORDS}:

\textbf{Variabilidade espaço-temporal da qualidade da água superficial do Corpo Central I da represa Billings - São Paulo, Brasil}

\textbf{RESUMO}:

\textbf{PALAVRAS-CHAVE}:

\hypertarget{introduction}{%
\subsection{INTRODUCTION}\label{introduction}}

\hypertarget{metologia}{%
\subsection{METOLOGIA}\label{metologia}}

\hypertarget{aspectos-uxe9ticos}{%
\subsubsection{Aspectos Éticos}\label{aspectos-uxe9ticos}}

Como parte do estudo aprovado pelo Comitê de Ética em Pesquisa Envolvendo Seres Humanos - CAEE 36189220.3.0000.8527 (``Perfil da população do oeste paranaense acometido de Síndrome Respiratória Agunda Grave entre 2020 a 2022'', Número do Parecer: 4.250.900), realizou-se uma análise comparativa das frequências alélicas em três grupos de pacientes com COVID-19 admitidos no Hospital Municipal Padre Germano Lauck - HMPGL (Foz do Iguaçu, Brasil). O estudo foi patrocinado pela Universidade Federal da Integração Latino-Americana (UNILA), como ``Ação 9 de enfrentamento à COVID-19'' (PORTARIA No 193/2020/GR) e pelo Conselho Nacional de Desenvolvimento Científico e Tecnológico - CNPq (Número).

\hypertarget{desenhos-do-estudo-e-participantes}{%
\subsubsection{Desenhos do Estudo e Participantes}\label{desenhos-do-estudo-e-participantes}}

Os pacientes com COVID-19 admitidos no HMPGL entre agosto à dezembro de 2020 e abril à junho de 2021 (semanas epidemiológicas) tiveram amostras de sangue coletadas. Os pacientes selecionados apresentaram durante a admissão resultado positivo para SARS-CoV-2 por transcrição reversa seguida de reação em cadeia de polimerase em tempo real (RT-qPCR) a partir de coleta nasofaríngea por swab ou lavagem broncoalveolar, realizado pelo Laboratório de Diagnóstico Molecular - HMPGL. Pacientes que apresentaram algum dos seguintes aspectos não foram incluídos nesse estudo: menores de dezoito anos, estrangeiros, admitidos por outras causas explicitamente não associadas à COVID-19 (i.e.~quedas, acidentes, lesões musculares), evasão hospitalar, resultado positivo após um significativo período de internação ou quadro prévio de imunosupressão adicional (i.e.~ativo no tratamento quimioterápico para câncer, imunodeficiente, HIV). Além disso, não foram incluídos pacientes de 2021 que foram readmitidos por quadro de COVID-19.
O perfil e quadro clínico de cada paciente foram coletados pelo prontuários do Sistema de Gestão Tasy (Koninklijke Philips N.V, Inc., Amsterdam, NL). As coletas de ambos os anos foram subdivididos baseados no desfecho: índividuos admitidos na UTI que receberam alta hospitalar (Alta-20 para amostragem de 2020, Alta-21 para amostragem de 2021) ou que faleceDesenhos do Estudo e Participantes

Os pacientes com COVID-19 admitidos no HMPGL entre agosto à dezembro de 2020 e abril à junho de 2021 (semanas epidemiológicas) tiveram amostras de sangue coletadas. Os pacientes selecionados apresentaram durante a admissão resultado positivo para SARS-CoV-2 por transcrição reversa seguida de reação em cadeia de polimerase em tempo real (RT-qPCR) a partir de coleta nasofaríngea por swab ou lavagem broncoalveolar, realizado pelo Laboratório de Diagnóstico Molecular - HMPGL. Pacientes que apresentaram algum dos seguintes aspectos não foram incluídos nesse estudo: menores de dezoito anos, estrangeiros, admitidos por outras causas explicitamente não associadas à COVID-19 (i.e.~quedas, acidentes, lesões musculares), evasão hospitalar, resultado positivo após um significativo período de internação ou quadro prévio de imunosupressão adicional (i.e.~ativo no tratamento quimioterápico para câncer, imunodeficiente, HIV). Além disso, não foram incluídos pacientes de 2021 que foram readmitidos por quadro de COVID-19.
O perfil e quadro clínico de cada paciente foram coletados pelo prontuários do Sistema de Gestão Tasy (Koninklijke Philips N.V, Inc., Amsterdam, NL). As coletas de ambos os anos foram subdividos baseados no desfecho: índividuos admitidos na UTI que receberam alta hospitalar (Alta-20 para amostragem de 2020, Alta-21 para amostragem de 2021) ou que faleceram (Óbito-20 para amostragem de 2020, Óbito-21 para amostragem de 2021). Durante a internação foram coletados cerca de 6 ml de sangue periférico em tubo (Tubo Greiner Bio-One) contendo o anticoagulante ácido etilenodiamino tetra-acético (EDTA), centrifugados e armazenados em refrigeração - 80 ͦ C.
O grupo controle para análises de frequências alélicas foi estabelecido com o Allele Frequency Net Database, a partir dos dados do Registro Nacional de Doadores de Médula Óssea - Paraná (REDOME-Paraná) (Gonzalez-Galarza et al., 2020).

\hypertarget{analysis}{%
\subsubsection{Analysis}\label{analysis}}

The analysis and visualization of data and statistical analysis were performed using the R software (\emph{\url{https://www.r-project.org/}}) (R Core Team, 2021). R Packages used in this research are available on the Comprehensive R Archive Network (CRAN).

\hypertarget{results-and-discussion}{%
\subsection{RESULTS AND DISCUSSION}\label{results-and-discussion}}

\hypertarget{references}{%
\subsection{REFERENCES}\label{references}}

\hypertarget{refs}{}
\begin{CSLReferences}{0}{0}
\leavevmode\vadjust pre{\hypertarget{ref-gonzalez-galarza_allele_2020}{}}%
GONZALEZ-GALARZA, F. F. et al. Allele frequency net database ({AFND}) 2020 update: gold-standard data classification, open access genotype data and new query tools. \textbf{Nucleic Acids Research}, v. 48, n. D1, p. D783--D788, jan. 2020. Disponível em: \textless{}\url{https://doi.org/10.1093/nar/gkz1029}\textgreater. Acesso em: 7 jun. 2022.\url{https://doi.org/10.1093/nar/gkz1029}.

\leavevmode\vadjust pre{\hypertarget{ref-R-software}{}}%
R CORE TEAM. \textbf{\href{https://www.R-project.org/}{R: A Language and Environment for Statistical Computing}}. Vienna, Austria: R Foundation for Statistical Computing, 2021.

\end{CSLReferences}

\end{document}
